\documentclass[12pt,svgnames]{article}
%
\input{C:/Raut/Workfile/Papers/TopBiblatex}
\bibliography{BibFile1.bib,BibFile2.bib}
%\bibliography{Rautlib,eBooks,Raut_Own}
\newtheorem{assn}{Assumption}
\input tcilatex

\begin{document}
\title{Tex2Rmd: A package to covert Latex document to Rmarkdown and then to html and word format\thanks{%
Many thanks for user comments.
}}
\author{Lakshmi K. Raut \\
Visiting Fellow, CEHD\\
University of Chicago\\
1126 E. 59th St., Chicago, IL 60637,USA
}
\date{\today}
\maketitle 
\begin{abstract}
The article describes a typical features of a Latex document that can be converted to the Rmarkdown format using this package. The converted markdown document, possibly after further addition of other R chunks, can be converted to html and word documents. It is also possible to convert back to Latex, but you may have to tweak it a little to make the latex output look nice. 

\textbf{Keywords:} Latex, html, Rmarkdown, Microsoft word. 
\end{abstract}

\section{Introduction}\label{sec1}
This article uses bits and pieces Latex contents from my own papers to illustrate features of this package. The binaries are given for windows (32bit should also work for 64bit and Linux 64bit). It is important to \emph{emphasize} in bold text \textbf{NOT ALL} features involving complex formatting codes can be converted.  It incorporates the basic minimum features that are generally used in a Ph.D. thesis, scientific article or a book. See this footnote\footnote{%
It does not convert latex tables completely. This verion of the software only creates a template for table with caption and label, you need to fill in the details in Rmarkdown document.} for limitations on converting Latex table environment.

Section \ref{sec2} discusses how it converts references. \autoref{sec3} shows what kind of equations both displayed and inline are converted. Section \ref{sec4} shows conversions of figures and tables. \autoref{sec5} describes the list like environments such as itemize, enumerate and description. These environments may be present inside other environments such as in proposition, theorem etc, as shown in \autoref{sec6}. Conversions of list environments are not perfect, you may need some tweaking in the converted Rmarkdown document. \autoref{sec6} shows theorem like Latex environments such as Theorem, Lemma, Proposition each with its own auto numbering and proof like environments without numbering. \autoref{sec7} discusses a few important facts and how to do remarks.



\section {Referencing}\label{sec2}
The main findings in \citep{Aalen.etal_2008_Book,Kanherkar.etal_2014} are that .... For the effects of early childhood factors on school and labor market outcomes, see \cite{Heckman.Raut_2016}, and also see \cite{Raut_2018} with an updated references. In machine learning, \cite{Altae-Tran_2016,Altae-Tran.etal_2017} show how an RNN can be used with limited data. 

\section{Equations}\label{sec3}

Here is an eqnarray environment copied directly from my paper, \cite{Raut_2019}. 
\begin{eqnarray}
\lambda _{hj}(t) &=&\lim_{\Delta t\rightarrow 0}\frac{P_{hj}(t,t+\Delta
t)-P_{hj}\left( t,t\right) }{\Delta t},\text{for }j\in S,\text{ which for }%
j\neq h\text{ becomes}  \notag \\
&=&\lim_{\Delta t\rightarrow 0}\frac{P_{hj}(t,t+\Delta t)}{\Delta t},\text{
and for }j=h\text{ becomes}  \notag \\
\lambda _{hh}(t) &=&\lim_{\Delta t\rightarrow 0}\frac{P_{hh}(t,t+\Delta t)-1%
}{\Delta t}  \label{eq3} \\
&=&-\lim_{\Delta t\rightarrow 0}\frac{\sum_{j\neq h}P_{hj}(t,t+\Delta t)}{%
\Delta t}  \notag \\
&=&-\sum_{j\neq h}\lambda _{hj}\left( t\right)   \notag
\end{eqnarray}%

See section \ref{sec6} for more equations.  You can have inline math such as $\int f(x) d\mu(x)$. 

\section{Figures and Tables} \label{sec4}

This is an example of converting a Latex figure with includegraphics in png format. It can be referred in the text using \autoref{fig1}.
 
\begin{figure}[htbp]
\begin{center}
\includegraphics[height=3.0in,keepaspectratio=true]{tree1.png}
\end{center}
\caption{Extensive form representation of the multi-stage game, $\Gamma(h_t)$}
\label{fig1}
\end{figure}

The software can also convert Tikz pictures as well.  Here is one taken from my \cite{Raut_2017a} paper.

\begin{figure}[tbp]
\begin{center}
\begin{tikzpicture}[scale=0.5]
\draw[thick,<->] (0,10) node[left]{$\tau_t$}--(0,0)--(10,0) node[below]{$s_{t-1}$};
\node [below left] at (0,0) {$0$};
\draw(0.5,10) ..controls (1,5) and (3.5,2) .. (10,0.5) node[right]{$\sigma(\tau_t, s_{t-1})=s_t$};
\draw(1,10) ..controls (2,4.5) and (3,3.5) .. (10,1.2) node[right]{$\sigma(\tau_t, s_{t-1})=s'_t$};
\node [below left] at (6,6) {$s'_t > s_t$};
\end{tikzpicture}
\end{center}
\caption{Sets of individuals $(\tau_t,s_{t-1})$ for whom $\sigma(\tau_t,s_{t-1})=s_t$ and $\sigma(\tau _t,s_{t-1})=s^{\prime }_t$}
\label{fig2}
\end{figure}

The following table can be referenced like "Table \ref{table2}" or like "\autoref{table2}". Both produce the same link as you can see.
%==============================================================================
\begin{table}[!h]
\footnotesize\centering
\caption{Steady-state local learning and subgame perfect gift equilibria for the economy with $\delta _{0}=0.35$}
\label{table2}\vspace {.10in}
\par
\begin{tabular}{||c|c|c|l||}
\hline\hline
$\tau $ & $%
\begin{array}{l}
\text{Equilibrium} \\ 
\text{Concept}%
\end{array}
$ & $\sigma ^{\prime *}$ & $\left( n^{*},s^{*},a^{*},U_{\max }\right) $ \\ 
\hline
0 & $%
\begin{array}{l}
\text{Nash} \\ 
\text{Equilibrium}%
\end{array}
$ & - & $%
\begin{array}{l}
(1.699710194,0,.4095616885,-1.140189766) \\ 
(1.025062190,1.341247016,.3341720874,-1.241803182)%
\end{array}
$ \\ \hline
- & $%
\begin{array}{l}
\text{Social} \\ 
\text{Optimum}%
\end{array}
$ & - & $n^{*}=4.4273139,$ $\tau ^{*}=.296681,$ $U_{\max }=-1.066475$ \\ 
\hline
0 & $%
\begin{array}{l}
\text{Fixed} \\ 
\text{Convention}%
\end{array}
$ & 573.2 & $(1.6958508998,0,0.409831247,-1.140547454)$ \\ \hline
0 & $%
\begin{array}{l}
\text{Fixed} \\ 
\text{Convention}%
\end{array}
$ & 0 & $%
\begin{array}{l}
(1.5989049725, 0, 0.41682122123, -1.1501342368) \\ 
(.8658794251, 1.477940857, .3265849827,-1.270158580)%
\end{array}
$ \\ \hline
0 & $%
\begin{array}{l}
\text{Learned} \\ 
\text{Convention}%
\end{array}
$ & -0.121472602158 & (1.59835683672,0, 0.4168619874, -1.1501919166) \\ 
\hline
0.035 & $%
\begin{array}{l}
\text{Learned} \\ 
\text{Convention}%
\end{array}
$ & -0.0884944056 & (1.4123165415, 0, 0.39660878 -1.1724263009) \\ 
\hline\hline
\end{tabular}
\vspace{0.25in}
\end{table}
%=================================================================

\section{List like environments: itemize, enumerate, description}\label{sec5}

Enumerate items

\begin{enumerate}
\item One
\item Two
\item Three
\end{enumerate}

Description items

\begin{description} 
  \item Can convert theorems and theorem like environments like proposition, lemma etc.

  \item Can convert proof environment. 
  
  \item Can convert assumption and remark environments: Assumption is user created environment, and remarks either as Latex environment or user created latex environment.
   
  \item Figures: Can convert includegraphics with png files and embedded Tikz figure environment. 
  
  \item Tables: It converts tables as a Rmarkdown table chunk keeping only the label and caption of the Latex tables. The references to the table is also converted throughout the document.  The content of a Latex table may involve complex structure and often created using excel or R and better left for various R packages to create those Rmarkdown table contents in the converted Rmarkdown document.
  
\end{description} 

\section{Theorem like environments} \label{sec6}
I illustrate the content of this section taking a section of my paper, \cite{Raut_2017a}. This involves definition, theorem and proof environments.  Similarly, it will convert other theorem and theorem like environments of your Latex document.
 
\begin{definition}
\label{def1} Initial distribution $\pi ^{0}$ of social groups in $\mathcal{S}$, is given. A \textbf{signaling equilibrium}\index{Signaling equilibrium} is a sequence of probability distributions $\left\{ q_{t}\left( e_{t}|s_{t}\right) \right\} _{1}^{\infty }$ and a sequence of optimal schooling decision rules $\left\{ \sigma _{t}\left( \tau _{t},s_{t-1}\right) \right\} _{1}^{\infty }$ such that at each period $t\geq 1,$

\begin{enumerate}
\item The induced wage schedule $w_{t}\left( s_t\right) =\int e_{t}q_{t}\left(
e_{t}|s_{t}\right)  de_{t}$ is a smooth concave function.

\item Given $w_{t}\left( s_t\right)$, the function $\sigma _{t}\left(\tau
_{t},s_{t-1}\right)$ solves the schooling decision problem of each
agent $\left( \tau _{t},s_{t-1}\right)$.

\item The induced  conditional distribution $\hat{q}_{t}\left( e_{t}|s_{t}\right)$ of $e_{t}$ given the optimal solution 
$s_{t}=\sigma _{t}\left( \tau _{t},s_{t-1}\right)$ obtained by using Bayes
rule coincides with the anticipated conditional distribution $q_{t}\left(
e_{t}|s_{t}\right)$ for all $s_{t}$.
\end{enumerate}
\end{definition}


I assume the following:

\begin{assumption}
\label{A1}$\theta _{t}(s_{t},\tau _{t},s_{t-1})$ $=\theta _{1}\left( s_{t}\right) \cdot \theta _{2}\left( \tau _{t}\right) \cdot \theta _{3}\left( s_{t-1}\right) ,$ $\theta _{1}\left( {}\right)$ is smooth, monotonically increasing and concave, $\theta _{2}\left( {}\right)$ and $\theta _{3}\left( .\right)$ are smooth, monotonically decreasing.
\end{assumption}

\begin{assumption}
\label{A2}The distributions $g\left( \tau \right)$ and $\pi _{0}\left( s_{0}\right)$ belong to a concave conjugate family.
\end{assumption}

\begin{theorem}
Under \autoref{A1} and \autoref{A2}, there exists a signaling equilibrium.
\end{theorem}

\begin{proof}
Suppose we have found a smooth concave wage schedule $w_{t}\left( s\right)$
with a first derivative $w_{t}^{\prime }\left( {}\right)$. The first order
condition of the optimization problem is given by 

\begin{equation}
\frac{w_{t}^{\prime }\left( s_{t}\right) }{\theta _{1}^{\prime }\left(
s_{t}\right) }=\theta _{2}\left( \tau _{t}\right) \theta _{3}\left(
s_{t-1}\right)  \label{eqq2}
\end{equation}

The rest is given in \cite{Raut_2017a}.

\end{proof}

\section{Remarks} \label{sec7}

Remarks are numbered and and can have labels which can used to refer to them in the text.


\begin{remark}
The processing of Rmarkdown file is best done in Rstudio. It can also be done in R. You need to have the following R packages in R or Rstudio, issuing command: install.packages(c("knitr","rmarkdown","bookdown","reticulate","pdftools","magick")). Package reticulate is needed if you want to incorporate python codes in Rmarkdown document. Apart from R, you need to have a Latex document processing system such as Miktex for windows and Tex Live for Linux.  You also need pandoc which is automatically installed with RStudio installation, otherwise you need this package. You also need to pandoc's pandoc-crossref package that can work with your pandoc package.  
\end{remark}


\begin{remark}
\label{re10}This remark can be referred in the text. In the latex document using its convention. To see how it is to be done in Rmarkdown, see the converted Rmarkdown document and the text below it.
\end{remark}


The bibliography also can be converted. It does not have to be the last statement. 

\printbibliography
\end{document}